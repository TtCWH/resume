% !TEX TS-program = xelatex
% !TEX encoding = UTF-8 Unicode
% !Mode:: "TeX:UTF-8"

\documentclass{resume}
\usepackage{zh_CN-Adobefonts_external} % Simplified Chinese Support using external fonts (./fonts/zh_CN-Adobe/)
%\usepackage{zh_CN-Adobefonts_internal} % Simplified Chinese Support using system fonts
\usepackage{linespacing_fix} % disable extra space before next section
\usepackage{cite}

\begin{document}
\pagenumbering{gobble} % suppress displaying page number

\name{汪寒}

\basicInfo{
  \email{wanghwh@foxmail.com} \textperiodcentered\ 
  \phone{(+86) 17794536556} \textperiodcentered\ 
  \linkedin[billryan8]{https://www.linkedin.com/in/billryan8}}
 
\section{\faGraduationCap\  教育背景}
\datedsubsection{\textbf{浙江大学}, 杭州,浙江}
\textit{在读硕士研究生}\ 计算机科学与技术, 预计 2020 年 3 月毕业
\datedsubsection{\textbf{江南大学}, 无锡, 江苏}
\textit{学士}\ 电子工程

\section{\faUsers\ 实习/项目经历}
\datedsubsection{\textbf{海知智能} 上海}{2017年5月 -- 2017年9月}
\role{实习}
知识图谱实习生
\begin{itemize}
  \item 数据爬取和数据处理
  \item 爬取了百度百科,天天基金网等数十个网站的数据
  \item 管理部分爬虫服务器并实现定时更新
\end{itemize}

\datedsubsection{\textbf{部分NLP论文复现}}
\role{Tensorflow, Python}{个人项目}
\begin{onehalfspacing}
自行复现了部分论文代码,并实现了相近的性能
\end{onehalfspacing}

% Reference Test
%\datedsubsection{\textbf{Paper Title\cite{zaharia2012resilient}}}{May. 2015}
%An xxx optimized for xxx\cite{verma2015large}
%\begin{itemize}
%  \item main contribution
%\end{itemize}

\section{\faCogs\ IT 技能}
% increase linespacing [parsep=0.5ex]
\begin{itemize}[parsep=0.5ex]
  \item 编程语言: Python > C++
  \item 平台: 可以较为熟练地使用Linux
  \item 开发: 较为了解Tensorflow框架的使用
\end{itemize}


%% Reference
%\newpage
%\bibliographystyle{IEEETran}
%\bibliography{mycite}
\end{document}